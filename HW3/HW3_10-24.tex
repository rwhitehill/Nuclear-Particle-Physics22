\def\duedate{10/24/22}
\def\HWnum{3}
% Document setup
\documentclass[12pt]{article}
\usepackage[margin=1in]{geometry}
\usepackage{fancyhdr}
\usepackage{lastpage}

\pagestyle{fancy}
\lhead{Richard Whitehill}
\chead{PHYS 675 -- HW \HWnum}
\rhead{\duedate}
\cfoot{\thepage \hspace{1pt} of \pageref{LastPage}}

% Encoding
\usepackage[utf8]{inputenc}
\usepackage[T1]{fontenc}

% Math/Physics Packages
\usepackage{amsmath}
\usepackage{mathtools}
\usepackage[arrowdel]{physics}
\usepackage{siunitx}

\AtBeginDocument{\RenewCommandCopy\qty\SI}

% Reference Style
\usepackage{hyperref}
\hypersetup{
    colorlinks=true,
    linkcolor=blue,
    filecolor=magenta,
    urlcolor=cyan,
    citecolor=green
}

\newcommand{\eref}[1]{Eq.~(\ref{eq:#1})}
\newcommand{\erefs}[2]{Eqs.~(\ref{eq:#1})--(\ref{eq:#2})}

\newcommand{\fref}[1]{Fig.~\ref{fig:#1}}
\newcommand{\frefs}[2]{Figs.~\ref{fig:#1}--\ref{fig:#2}}

\newcommand{\tref}[1]{Table~\ref{tab:#1}}
\newcommand{\trefs}[2]{Tables~\ref{tab:#1}-\ref{tab:#2}}

% Figures and Tables 
\usepackage{graphicx}
\usepackage{float}
\usepackage{booktabs}

\newcommand{\bef}{\begin{figure}[h!]\begin{center}}
\newcommand{\eef}{\end{center}\end{figure}}

\newcommand{\bet}{\begin{table}[h!]\begin{center}}
\newcommand{\eet}{\end{center}\end{table}}

% tikz
\usepackage{tikz}
\usetikzlibrary{calc}
\usetikzlibrary{decorations.pathmorphing}
\usetikzlibrary{decorations.markings}
\usetikzlibrary{arrows.meta}
\usetikzlibrary{positioning}

% tcolorbox
\usepackage[most]{tcolorbox}
\usepackage{xcolor}
\usepackage{xifthen}
\usepackage{parskip}

\newcommand*{\eqbox}{\tcboxmath[
    enhanced,
    colback=black!10!white,
    colframe=black,
    sharp corners,
    size=fbox,
    boxsep=8pt,
    boxrule=1pt
]}

% Miscellaneous Definitions/Settings
\newcommand{\prob}[2]{\textbf{#1)} #2}

\setlength{\parskip}{\baselineskip}
\setlength{\parindent}{0pt}
\setlength{\headheight}{14.49998pt}
\addtolength{\topmargin}{-2.49998pt}


\begin{document}
    
\prob{1}{}

a) What are the possible $J^{PC}$ (total angular momentum, parity, and charge quantum numbers) values for mesons made up of a quark and an antiquark of the same flavor?
Enumerate all the possibilities through $J = 4$.

The total angular momentum $J = L + S$, where $L$ is the orbital angular momentum and $S$ is the spin angular momentum. 
Since the meson system ($q\overline{q}$) is composed of 2 fermions, it must be a boson, meaning it can only be spin-0 or spin-1.
The total angular momentum for a given $s,\ell$ values for spin and orbital angular momentum $j \in \{ |s - \ell|, \ldots, s + \ell - 1, s + \ell \} $.
The parity quantum number is multiplicative, with $P(\overline{q}) = - P(q)$ and $|P(q)| = 1$.
Thus, $P(q\overline{q}) = -[P(q)]^2(-1)^{\ell} = (-1)^{\ell + 1}$.
Additionally, the charge quantum number takes on the value $(-1)^{\ell + s}$ for a system consiting of a particle-antiparticle pair.

\begin{table}[H]
\begin{center}
\begin{tabular}{ccccc}
    S & L & J & P & C\\
    \hline
    0 & 0 & 0 & - & + \\
    1 & 0 & 1 & - & - \\
    0 & 1 & 1 & + & - \\
    1 & 1 & $\{ 0,1,2 \} $ & + & + \\
    0 & 2 & 2 & - & + \\
    1 & 2 & $\{ 1,2,3 \} $ & - & - \\
    0 & 3 & 3 & + & - \\
    1 & 3 & $\{ 2,3,4 \} $ & + & + \\
    0 & 4 & 4 & - & +
\end{tabular}
\end{center}
\caption{Table of possible $J$ and $P$ values for unflavored mesons given combinations of $S$ and $L$ values.}
\label{tab:JPC-values}
\end{table}

These combinations give us the following set of possible $J^{PC}$ values:
\begin{align}
    \label{eq:JPC}
    J^{PC} &\in \{ 0^{-+}, 1^{--}, 1^{+-}, 0^{++}, 1^{++}, 2^{++}, 2^{-+}, 2^{--}, 3^{--}, 3^{+-}, 3^{++}, 4^{++}, 4^{-+} \}
.\end{align}

b) Which of these values can be found for mesons in the PDG listings? 

The PDG lists the following values of $J^{PC}$ in the light unflavored meson section, which correspond with the list in \eref{JPC}:
\begin{align}
    \label{eq:JPC-PDG}
    \{ 0^{-+}, 0^{++}, 1^{--}, 1^{+-}, 1^{++}, 2^{++}, 1^{--}, 2^{-+}, 3^{--}, 4^{++} \} 
.\end{align}


c) What $J^{PC}$ values are not possible for a quark-antiquark pair?
Can any of them be found in the PDG listings?

The following $J^{PC}$ values are listed in the PDG, which are not listed in \eref{JPC}:
\begin{eqnarray}
    \label{eq:JPC-forbidden}
    \{ 1^{-+} \}  
.\end{eqnarray}
Clearly, this indicates that our model for mesons simply as $q\overline{q}$ pairs is insufficient and incorrect.

d) What $J^{PC}$ values are possible for a meson that decays strongly into $\pi^{+}\pi^{-}$?

The $\pi^{\pm}$ have $J = 0$ and $P = -1$ separately.
As a system $P = (-1)^{\ell}$, and $J = \ell$.
In any strong decay, parity and spin is conserved so it follows that the possible values of $J^{PC} = 0^{+\pm}$.

e) What $J^{PC}$ values are possible for a meson that decays strongly into $\pi^{0}\pi^{0}$?
What are the possible isospin values?

We know $J^{PC} = 0^{-+}$ for the $\pi^{0}$.
As a system $P = -1$ and $C = -1$, so we need $J = 0$ and $J^{PC} = 0^{--}$.

From isospin considerations $\pi^{0} = \ket{1,0}$, and using the Clebsch-Gordan coefficients, we can write
\begin{eqnarray}
    \label{eq:isospin-CG}
    \ket{1,0}\ket{1,0} = \sqrt{\frac{2}{3}}\ket{2,0} - \sqrt{\frac{1}{3}}\ket{0,0}
,\end{eqnarray}
so the initial meson could have isospin $0$ or $2$.

f) If you discovered a spin-0 meson that decays strongly into $\rho^{0}\pi^{0}$, why would you want to rush this result off for immediate publication?

We strongly believe that spin is a conserved quantity in a decay or scattering reaction.
The initial state has spin-0 (as given), but the final state has spin-1 ($s_{\rho} = 1$ and $s_{\pi} = 1$).
This would be a monumental finding if true since it would be the first observed instance of spin conservation being violated. 

\prob{2}{
Show that the $\eta(548)$ meson cannot decay into 2 pions by either the strong or electromagnetic interactions and that it cannot decay into 3 pions by the strong interactions.
}

The $\eta$ meson has $J^{PC} = 0^{-+}$ and is (iso)spin-0, where the pions have $J = 0$ and $P = -1$ ($C = +1$ for the $\pi^{0}$).
For a two pion system, we have $P = 1$, so it is clear that an $\eta$ decay into two pions is impossible electromagnetically and strongly, which both conserve parity.
Notice that for a three pion system, $G = -1$ but $G = +1$ for the $\eta$ meson, and since G-parity is conserved in a strong interaction, the $\eta$ cannot decay via the strong interaction into $3$ pions.



\prob{3}{
Write an equation relating the cross sections for these four reactions:
\begin{eqnarray}
    \label{eq:reactions}
    \begin{cases}
    \overline{K}^{0}p \rightarrow \Sigma^{0}\pi^{+} \\
    \overline{K}^{0}n \rightarrow \Sigma^{+}\pi^{-} \\
    \overline{K}^{0}n \rightarrow \Sigma^{0}\pi^{0} \\
    \overline{K}^{0}n \rightarrow \Sigma^{-}\pi^{+}
    \end{cases}
.\end{eqnarray}

}

\prob{4}{
    As an example of the use of isospin in weak decays, consider the decays $\tau^{+} \rightarrow \pi^{+}\pi^{+}\pi^{-}\overline{\nu}_{\tau}$ and $\tau^{+} \rightarrow \pi^{+}\pi^{0}\pi^{0}\overline{\nu}_{\tau}$.
}

a) Draw the diagram for these decays at the quark level (before hadronization) and observe that there is only one possible isospin state.

b) There is strong evidence that the three pions in these decays come from a $\rho \pi$ intermediate state.
Assuming that, predict the ratio of branching fractions for these two decays.
Compare to data (PDG).



\prob{5}{
The neutral vector mesons all have measurable decays into $e^{+}e^{-}$.
%The diagram for this decay is shown on the right.
Since the quark and antiquark are in a bound state, the calculation of these decay rates depends on the vector meson wave functions and masses.
However, it is reasonable to assume that these are all about the same for the three light neutral vector mesons, the $\rho^{0}$, the $\omega$, and the $\phi$.
Make this assumption and predict the relative partial widths.
}




\end{document}
