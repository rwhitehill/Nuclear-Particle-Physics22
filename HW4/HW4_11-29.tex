\def\duedate{11/29/22}
\def\HWnum{4}
% Document setup
\documentclass[12pt]{article}
\usepackage[margin=1in]{geometry}
\usepackage{fancyhdr}
\usepackage{lastpage}

\pagestyle{fancy}
\lhead{Richard Whitehill}
\chead{PHYS 675 -- HW \HWnum}
\rhead{\duedate}
\cfoot{\thepage \hspace{1pt} of \pageref{LastPage}}

% Encoding
\usepackage[utf8]{inputenc}
\usepackage[T1]{fontenc}

% Math/Physics Packages
\usepackage{amsmath}
\usepackage{mathtools}
\usepackage[arrowdel]{physics}
\usepackage{siunitx}

\AtBeginDocument{\RenewCommandCopy\qty\SI}

% Reference Style
\usepackage{hyperref}
\hypersetup{
    colorlinks=true,
    linkcolor=blue,
    filecolor=magenta,
    urlcolor=cyan,
    citecolor=green
}

\newcommand{\eref}[1]{Eq.~(\ref{eq:#1})}
\newcommand{\erefs}[2]{Eqs.~(\ref{eq:#1})--(\ref{eq:#2})}

\newcommand{\fref}[1]{Fig.~\ref{fig:#1}}
\newcommand{\frefs}[2]{Figs.~\ref{fig:#1}--\ref{fig:#2}}

\newcommand{\tref}[1]{Table~\ref{tab:#1}}
\newcommand{\trefs}[2]{Tables~\ref{tab:#1}-\ref{tab:#2}}

% Figures and Tables 
\usepackage{graphicx}
\usepackage{float}
\usepackage{booktabs}

\newcommand{\bef}{\begin{figure}[h!]\begin{center}}
\newcommand{\eef}{\end{center}\end{figure}}

\newcommand{\bet}{\begin{table}[h!]\begin{center}}
\newcommand{\eet}{\end{center}\end{table}}

% tikz
\usepackage{tikz}
\usetikzlibrary{calc}
\usetikzlibrary{decorations.pathmorphing}
\usetikzlibrary{decorations.markings}
\usetikzlibrary{arrows.meta}
\usetikzlibrary{positioning}

% tcolorbox
\usepackage[most]{tcolorbox}
\usepackage{xcolor}
\usepackage{xifthen}
\usepackage{parskip}

\newcommand*{\eqbox}{\tcboxmath[
    enhanced,
    colback=black!10!white,
    colframe=black,
    sharp corners,
    size=fbox,
    boxsep=8pt,
    boxrule=1pt
]}

% Miscellaneous Definitions/Settings
\newcommand{\prob}[2]{\textbf{#1)} #2}

\setlength{\parskip}{\baselineskip}
\setlength{\parindent}{0pt}
\setlength{\headheight}{14.49998pt}
\addtolength{\topmargin}{-2.49998pt}

\usepackage[]{slashed}

\begin{document}
    
\prob{1}{
In the toy model introduced in lecture ($A$, $B$, and $C$ only)
}

a) Calculate the total cross section for $A \bar{A} \rightarrow A \bar{A}$ for the case $m_{A} = 0$, $m_{C} \gg E_{A}$.
Remember there are two diagrams for this process.

The tree level diagrams are shown in the following image, with momenta labeled:

For diagram 1, we have
\begin{eqnarray}
    \label{eq:amp1}
    \mathcal{M}_{1} = i(-ig)\frac{i}{s - m_{C}^2}(-ig) 
,\end{eqnarray}
where $s = (p_1 + p_2)^2$,
and for diagram 2 we have
\begin{eqnarray}
    \label{eq:amp2}
    \mathcal{M}_{2} = i(-ig)\frac{i}{t - m_{C}^2}(-ig) 
,\end{eqnarray}
where $t = (p_1 - p_3)^2$.
The total amplitude is given as
\begin{eqnarray}
    \label{eq:squared-amp}
    \mathcal{M} = \frac{g^{2}}{s - m_{C}^2} + \frac{g^{2}}{t - m_{C}^2}
.\end{eqnarray}
We can simplify the Mandelstam variables ($s$ and $t$) as 
\begin{eqnarray}
    \label{eq:mand-vars}
    \begin{aligned}    
        s &= p_1^2 + p_2^2 + 2 p_1 \cdot p_2 = \phantom{-} 2 p_1 \cdot p_2 = 4 E_{A}^2 \\
        t &= p_1^2 + p_3^2 - 2 p_1 \cdot p_3 = -2 p_1 \cdot p_3 = -2 E_{A}^2 (1 + \cos{\theta})
    .\end{aligned}
\end{eqnarray}
Note that $E_{A} = E_{\bar{A}} = | \mathbf{p} |$ since $m_{A} = m_{\bar{A}} = 0$.
The center of mass differential cross section is then
\begin{align}
    \label{eq:dxsec-CM}
    \dv{\sigma}{\Omega} &= \frac{1}{64 \pi^2 (E_{A} + E_{\bar{A}})^2} \frac{| \mathbf{p}' |}{| \mathbf{p} |} | \mathcal{M} |^2 \\
                        &= \frac{g^{4}}{64 \pi^2 (4 E_{A}^2)} \Big[ \frac{1}{s - m_{C}^2} + \frac{1}{t - m_{C}^2} \Big]^2 \\
                        &= \frac{g^{4}}{64 \pi ^2 E_{A}^2 m_{C}^{4}} \Big[ 1 + \Big( \frac{E_{A}}{m_{C}} \Big)^2 \sin^2{(\theta/2)} \Big]
,\end{align}
where $E_{A}$ and $E_{\bar{A}}$ are the initial state energies of the corresponding particles and $\mathbf{p}$ and $\mathbf{p}'$ are the initial and final state 3-momenta in the CM frame.
Note that the term in square brackets was expanded in a series of powers of $E_{A}/m_{C}$ and higher order terms were thrown away.
Integrating over the solid angle $\Omega$, we find the total cross section to be
\begin{eqnarray}
    \label{eq:tot-xsec-AAbar}
    \eqbox{
        \sigma = \frac{g^{4}}{16 \pi E_{A}^2 m_{C}^{4}} \Bigg[ 1 + \frac{1}{2} \Big( \frac{E_{A}}{2 m_{C}} \Big)^2 \Bigg]
}
.\end{eqnarray}


b) Calculate the total cross section for $A A \rightarrow A A$ for the case $m_{A} = 0$, $m_{C} \gg E_{A}$.
Remember that there are two diagrams for this process and that the integral should be taken only over half the solid angle or the $S$ factor should be used to avoid double counting.
What is the ratio of the cross sections for parts (a) and (b)?

The two possible diagrams are shown below:

As in part (a) we use our toy Feynman rules to find that
\begin{align}
    \label{eq:amps-b}
    \mathcal{M}_{1} &= \frac{g^2}{(p_1 - p_3)^2 - m_{C}^2} \\
    \mathcal{M}_{2} &= \frac{g^2}{(p_3 - p_2)^2 - m_{C}^2}
,\end{align}
which gives the total amplitude as
\begin{eqnarray}
    \label{eq:amp-b}
    \mathcal{M} = g^2 \Big[ \frac{1}{t - m_{C}^2} + \frac{1}{u - m_{C}^2} \Big]
\end{eqnarray}
and differential cross section as
\begin{eqnarray}
    \label{eq:dxsec-b}
    \dv{\sigma}{\Omega} = \frac{g^{4}}{128 \pi^2 E_{A}^2 m_{C}^{4}} \Bigg[ 1 - \Big( \frac{2E_{A}}{m_{C}} \Big)^2 \Bigg]
.\end{eqnarray}
We then obtain the total cross section by integrating over the solid angle:
\begin{eqnarray}
    \label{eq:xsec-b}
    \eqbox{
        \sigma = \frac{g^{4}}{32 \pi E_{A}^2 m_{C}^{4}} \Bigg[ 1 - \Big( \frac{2 E_{A}}{m_{C}} \Big)^2 \Bigg]
    } 
.\end{eqnarray}



c) Calculate the differential cross sections for $A \bar{A} \rightarrow A \bar{A}$ and $A A \rightarrow A A$ for the case $m_{A} = m_{C} = 0$.

Using our results for the amplitudes from parts (a) and (b) and inserting $m_{C} = 0$ we have
\begin{eqnarray}
    \label{eq:xsec1}
    \eqbox{
    \dv{\sigma(A \bar{A} \rightarrow A \bar{A})}{\Omega} = \frac{g^{4}}{64 \pi^2 (4 E_{A}^2)} \Big( \frac{1}{s} + \frac{1}{t} \Big)^2 = \Big( \frac{g^2 \tan^2{(\theta/2)}}{64 \pi E_{A}^3} \Big)^2
}
\end{eqnarray}
and
\begin{eqnarray}
    \label{eq:xsec2}
    \eqbox{
    \dv{\sigma(AA \rightarrow AA)}{\Omega} = \frac{g^{4}}{128 \pi^2 (4 E_{A}^2)} \Big( \frac{1}{t} + \frac{1}{u} \Big)^2 = \frac{1}{2} \Big( \frac{g^{2}}{16 \pi E_{A}^3 \sin^2{\theta}} \Big) ^2
}
.\end{eqnarray}


d) What is the ratio of the two cross sections for part (c) at very small angles?
What is the ratio for scattering at $90^{\circ}$ in the center of mass?
At $180^{\circ}$?

The ratio of the two cross sections is as follows:
\begin{eqnarray}
    \label{eq:ratio}
    R = \frac{\dd{\sigma(A \bar{A} \rightarrow A \bar{A})}}{\dd{\sigma(AA \rightarrow AA)}} = \frac{1}{8} \sin^{4}{\theta}\tan^{4}{(\theta/2)}
.\end{eqnarray}
At small angles, the ratio is
\begin{eqnarray}
    \label{eq:ratio-small-angle}
    R(\theta) = \frac{\theta^{8}}{128}
,\end{eqnarray}
which is incredibly small.

Furthermore,
\begin{eqnarray}
    \label{eq:ratio-90}
    R(90^{\circ}) = \frac{1}{8} 
,\end{eqnarray}
and
\begin{eqnarray}
    \label{eq:ratio-180}
    R(180^{\circ}) = \infty
.\end{eqnarray}



\prob{2}{
The Higgs boson has been discovered, cannot decay into two real $W$ or $Z$ pairs because its mass is too low.
However, consider another HIggs boson with a sufficiently high mass and spin 0.
Its coupling will be proportional to the Higgs mass divided by the $W$ mass, i.e. the vertex factor will be proportional to $-i g M_{H}/M_{W}$, where $g$ is a dimensionless number.
From dimensional analysis, how do you expect the Higgs decay width to vary with the unknown Higgs mass in this very high mass region?
}

\prob{3}{
Show that $S^{\dagger}S \ne 1$ but $S^{\dagger}\gamma^{0}S = \gamma^{0}$.
}

Recall that 
\begin{eqnarray}
    \label{eq:S-mat}
    S = \begin{pmatrix}
        a_{+} & a_{-}\sigma_1 \\
        a_{-}\sigma_1 & a_{+}
    \end{pmatrix}   
,\end{eqnarray}
where $a_{\pm} = \pm \sqrt{\frac{1}{2}(\gamma \pm 1)}$ and $\gamma = 1/\sqrt{1 - v^2}$ is the Lorentz boost factor (in natural units).
Thus,
\begin{eqnarray}
    \label{eq:SdaggerS}
    \begin{aligned}    
        S^{\dagger} S &= \begin{pmatrix}
        a_{+} & a_{-} \sigma_1 \\
        a_{-} \sigma_1 & a_{+}
    \end{pmatrix}
    \begin{pmatrix}
        a_{+} & a_{-} \sigma_1 \\
        a_{-} \sigma_1 & a_{+}
    \end{pmatrix}
    = 
    \begin{pmatrix}
        a_{+}^2 + a_{-}^2 \sigma_1^2 & 2 a_{+}a_{-} \sigma_1 \\
        2 a_{+}a_{-} \sigma_1 & a_{-}^2 \sigma_1^2 + a_{+}^2
    \end{pmatrix} 
    \\
                      &=
    \begin{pmatrix}
        \gamma & -\sqrt{\gamma^2 - 1} \sigma_1 \\
        -\sqrt{\gamma^2 - 1}\sigma_1 & \gamma
    \end{pmatrix}
    =
    \gamma
    \begin{pmatrix}
        1 & -v \sigma_1 \\
        -v \sigma_1 & 1
    \end{pmatrix} 
    \ne 1
    \end{aligned}
.\end{eqnarray}

Checking the second identity, we have
\begin{eqnarray}
    \label{eq:Sdagger-gamma0-S}
    S^{\dagger} \gamma^{0} S = 
    \begin{pmatrix}
        a_{+} & a_{-} \sigma_1 \\
        a_{-} \sigma_1 & a_{+}
    \end{pmatrix}
    \begin{pmatrix}
        1 & 0 \\
        0 & 1
    \end{pmatrix}
    \begin{pmatrix}
        a_{+} & a_{-} \sigma_1 \\
        a_{-} \sigma_1 & a_{+}
    \end{pmatrix}
    =
    \begin{pmatrix}
    a_{+}^2 - a_{-}^2\sigma_1^2 & 0 \\
    0 & -a_{+}^2 + a_{-}^2\sigma_1^2
    \end{pmatrix} 
    = \gamma^{0}
.\end{eqnarray}



\prob{4}{
Using the plane-wave solutions of the Dirac equation, verify the completeness relation 
\begin{eqnarray}
    \label{eq:completeness-spinors}
    \sum_{s=1,2} u^{(s)} \overline{u}^{(s)} = \slashed{p} + m
,\end{eqnarray}
where $\slashed{p} = \gamma^{\mu}p_{\mu}$ and natural units are used.
}

The spinors
\begin{eqnarray}
    \label{eq:particle-spinors}
    u^{(1)} = \sqrt{p^{0} + m}
    \begin{pmatrix}
        \chi_1 \\ \frac{ \boldsymbol{\sigma} \cdot \mathbf{p} }{p^{0} + m} \chi_1
    \end{pmatrix} 
    \quad
    u^{(2)} = 
    \sqrt{p^{0} + m}
    \begin{pmatrix}
        \chi_2 \\ \frac{ \boldsymbol{\sigma} \cdot \mathbf{p} }{p^{0} + m} \chi_2
    \end{pmatrix} 
,\end{eqnarray}
where $\displaystyle \chi_1 = \begin{pmatrix} 1 \\ 0 \end{pmatrix}$ and $\displaystyle \chi_2 = \begin{pmatrix} 0 \\ 1 \end{pmatrix}$.
For a generic spinor $u^{(s)}$
\begin{align}
    \label{eq:completeness-work}
    u^{(s)} \overline{u}^{(s)} &= (p^{0} + m) 
    \begin{pmatrix}
        \chi_{s} \\ \frac{ \boldsymbol{\sigma} \cdot \mathbf{p} }{p^{0} + m} \chi_{s}
    \end{pmatrix}
    \begin{pmatrix}
        \chi_{s}^{\dagger} & -\frac{ \boldsymbol{\sigma} \cdot \mathbf{p} }{p^{0} + m} \chi_{s}^{\dagger}
    \end{pmatrix} \notag \\
                               &= (p^{0} + m) 
                               \begin{pmatrix}
                                   \chi_{s} \chi_{s}^{\dagger} & - \frac{ \boldsymbol{\sigma} \cdot \mathbf{p} }{p^{0} + m} \chi_{s} \chi_{s}^{\dagger} \\
                                   \frac{ \boldsymbol{\sigma} \cdot \mathbf{p} }{p^{0} + m} \chi_{s} \chi_{s}^{\dagger} & - \frac{ (\boldsymbol{\sigma} \cdot \mathbf{p})^2 }{(p^{0} + m)^2} \chi_{s} \chi_{s}^{\dagger}
                               \end{pmatrix} 
.\end{align}
Noting that $\sum_{s=1,2} \chi_{s}\chi_{s}^{\dagger} = 1$, we have
\begin{align}
    \label{eq:completeness-work-2}
    \sum_{s=1,2} u^{(s)} \overline{u}^{(s)} &= (p^{0} + m)
    \begin{pmatrix}
        1 & -\frac{ \boldsymbol{\sigma} \cdot \mathbf{p} }{p^{0} + m} \\
        \frac{ \boldsymbol{\sigma} \cdot \mathbf{p} }{p^{0} + m} & -\frac{ \mathbf{p}^2 }{(p^{0} + m)^2}
    \end{pmatrix} 
    =
    (p^{0} + m)
    \begin{pmatrix}
        1 & -\frac{ \boldsymbol{\sigma} \cdot \mathbf{p} }{p^{0} + m} \\
        \frac{ \boldsymbol{\sigma} \cdot \mathbf{p} }{p^{0} + m} & - \frac{p^{0} - m}{p^{0} + m}
    \end{pmatrix} \notag \\
                                            &=
                                            \begin{pmatrix}
                                                p^{0} + m & 0 \\
                                                0 & -p^{0} + m
                                            \end{pmatrix}
                                            +
                                            \begin{pmatrix}
                                                0 & \boldsymbol{\sigma} \cdot \mathbf{p} \\
                                                - \boldsymbol{\sigma} \cdot \mathbf{p} & 0
                                            \end{pmatrix}
.\end{align}
Rearranging gives
\begin{eqnarray}
    \label{eq:completeness-result}
    \eqbox{
        \sum_{s=1,2} u^{(s)} \overline{u}^{(s)} = \gamma_0 p^{0} + \boldsymbol{\gamma} \cdot \mathbf{p} + m = \gamma_{\mu}p^{\mu} + m = \slashed{p} + m
}
.\end{eqnarray}



\prob{5}{
In class we calculated the differential cross section for $e^{+}e^{-} \rightarrow \mu^{+}\mu^{-}$ and obtained 
\begin{eqnarray}
    \label{eq:e+e-mu+mu-_xsec}
    \dv{\sigma}{\Omega} = \frac{\alpha^2}{4s}(1 + \cos^2{\theta})
,\end{eqnarray}
where $s$ is the total energy squared in the center of mass, $\theta$ is the angle between the $e^{-}$ and $\mu^{-}$ in the center of mass, and where we ignored the electron and muon masses.
Redo the calculuation for $e^{+}e^{-} \rightarrow \tau^{+}\tau^{-}$ for an energy range in which the electron mass can be ignored, but not the $\tau$ mass .
In particular, what is the factor by which the \textit{total} cross section is suppressed by the $\tau$ mass for the case in which the colliding electron and positron each have an energy of \SI{2}{\GeV}.
}

We worked out the unpolarized squared amplitude for the process $2 ~ \mbox{leptons} \rightarrow 2~\mbox{leptons}$ to be
\begin{eqnarray}
    \label{eq:lep-ampsq}
    \Big| \sum_{\mbox{\tiny spin}} \mathcal{M}_{i} \Big|^2 = \frac{g^{4}}{4s} L_{\mu\nu}^{e} L_{\tau}^{\mu\nu}
,\end{eqnarray}
where the unpolarized leptonic tensor for an arbitrary lepton appears generically as 
\begin{align}
    \label{eq:lep-tensor}
    L_{\mu\nu} &= \Tr\big[ \gamma^{\mu}(\slashed{p} + m)\gamma^{\nu}(\slashed{k} + m) \big] \notag \\
               &= \Tr\big[ \gamma^{\mu}\slashed{p}\gamma^{\nu}\slashed{k} \big] + m \Big[ \Tr\big[ \gamma^{\mu}\slashed{p}\gamma^{\nu} \big] + \Tr\big[ \gamma^{\mu}\gamma^{\nu}\slashed{k} \big] \Big] + m^2 \Tr\big[ \gamma^{\mu}\gamma^{\nu} \big] \notag \\
               &= 4( p^{\mu}k^{\nu} + k^{\mu}p^{\nu} - (p \cdot k - m^2) g^{\mu\nu} )
,\end{align}
where $p$ and $k$ are the incoming and outgoing momenta of the lepton leg.
In the kinematic regime specified in the problem statement we have that
\begin{align}
    \label{eq:lep-ampsq-2}
    \Big| \sum_{\mbox{\tiny spin}} \mathcal{M}_{i} \Big|^2 =& \frac{4g^{4}}{s} ( p_{e^{-}}^{\mu} p_{e^{+}}^{\nu} + p_{e^{-}}^{\mu} p_{e^{+}}^{\nu} - p_{e^{-}} \cdot p_{e^{+}} g^{\mu\nu} ) \times \notag \\
    &( p^{\tau^{-}}_{\mu} p^{\tau^{+}}_{\nu} + p^{\tau^{-}}_{\mu} p^{\tau^{+}}_{\nu} - (p_{\tau^{-}} \cdot p_{\tau^{+}} - m_{\tau}^2) g^{\mu\nu} ) \notag \\
    &= \frac{8g^{4}}{s} \big[ (p_{e^{-}} \cdot p_{\tau^{+}})(p_{e^{+}} \cdot p_{\tau^{-}}) + (p_{e^{-}} \cdot p_{\tau^{-}})(p_{e^{+}} \cdot p_{\tau^{+}}) - ( p_{e^{-}} \cdot p_{e^{+}} ) m_{\tau}^2 \big] \notag \\
    &= \frac{8g^{4}}{s} \big[ ( E E' + p p' \cos{\theta} )^2 + ( E E' - p p'\cos{\theta} )^2  - m_{\tau}^2( E^2 + p^2 ) \big]
.\end{align}
Note that from the conservation of energy, we must have $E = E'$ and $p' = \sqrt{E^2 -  m_{\tau}^2}$, where $p = E$ since we neglect the electron mass.
Expanding, cancelling, and simplifying we find
\begin{eqnarray}
    \label{eq:lep-ampsq-3}
    \Big| \sum_{\mbox{\tiny spin}} \mathcal{M}_{i} \Big|^2 = \frac{16 (4\pi)^2 \alpha^2 E^2}{s} (E^2 - m_{\tau}^2) (1 + \cos^2{\theta})
,\end{eqnarray}
where we used the fact that the coupling constant $g = e = \sqrt{4 \pi \alpha}$, so $g^{4} = (4 \pi)^2 \alpha^2$.
Thus, in the center of mass frame, the cross section is
\begin{align}
    \label{eq:xsec-etau}
    \dv{\sigma}{\Omega} &= \frac{1}{4} \frac{1}{64 \pi^2 (4E^2)} \frac{\sqrt{E^2 - m_{\tau}^2}}{E} \frac{256 \pi^2 \alpha^2 E^2}{s} (E^2 - m_{\tau}^2)(1 + \cos^2{\theta}) \notag \\
                        &= \eqbox{ \frac{\alpha^2}{4s} (1 + \cos^2{\theta}) \Big[ 1 - \Big(\frac{m_{\tau}}{E} \Big)^2 \Big]^{3/2} }
.\end{align}
The suppression factor is the last factor in the cross section.

If we have $E = \SI{2}{\GeV}$ then the suppression factor is
\begin{eqnarray}
    \label{eq:suppression-factor}
    \Big[ 1 - \Big( \frac{m_{\tau}}{E} \Big)^2 \Big]^{3/2} = 0.0967
,\end{eqnarray}
which means that the $e^{+}e^{-} \rightarrow \tau^{+}\tau^{-}$ cross section is only about 10\% of the $e^{+}e^{-} \rightarrow \mu^{+}\mu^{-}$ cross section.





\end{document}
