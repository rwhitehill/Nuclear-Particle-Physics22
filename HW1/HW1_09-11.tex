\def\duedate{09/11/22}
\def\HWnum{1}
% Document setup
\documentclass[12pt]{article}
\usepackage[margin=1in]{geometry}
\usepackage{fancyhdr}
\usepackage{lastpage}

\pagestyle{fancy}
\lhead{Richard Whitehill}
\chead{PHYS 675 -- HW \HWnum}
\rhead{\duedate}
\cfoot{\thepage \hspace{1pt} of \pageref{LastPage}}

% Encoding
\usepackage[utf8]{inputenc}
\usepackage[T1]{fontenc}

% Math/Physics Packages
\usepackage{amsmath}
\usepackage{mathtools}
\usepackage[arrowdel]{physics}
\usepackage{siunitx}

\AtBeginDocument{\RenewCommandCopy\qty\SI}

% Reference Style
\usepackage{hyperref}
\hypersetup{
    colorlinks=true,
    linkcolor=blue,
    filecolor=magenta,
    urlcolor=cyan,
    citecolor=green
}

\newcommand{\eref}[1]{Eq.~(\ref{eq:#1})}
\newcommand{\erefs}[2]{Eqs.~(\ref{eq:#1})--(\ref{eq:#2})}

\newcommand{\fref}[1]{Fig.~\ref{fig:#1}}
\newcommand{\frefs}[2]{Figs.~\ref{fig:#1}--\ref{fig:#2}}

\newcommand{\tref}[1]{Table~\ref{tab:#1}}
\newcommand{\trefs}[2]{Tables~\ref{tab:#1}-\ref{tab:#2}}

% Figures and Tables 
\usepackage{graphicx}
\usepackage{float}
\usepackage{booktabs}

\newcommand{\bef}{\begin{figure}[h!]\begin{center}}
\newcommand{\eef}{\end{center}\end{figure}}

\newcommand{\bet}{\begin{table}[h!]\begin{center}}
\newcommand{\eet}{\end{center}\end{table}}

% tikz
\usepackage{tikz}
\usetikzlibrary{calc}
\usetikzlibrary{decorations.pathmorphing}
\usetikzlibrary{decorations.markings}
\usetikzlibrary{arrows.meta}
\usetikzlibrary{positioning}

% tcolorbox
\usepackage[most]{tcolorbox}
\usepackage{xcolor}
\usepackage{xifthen}
\usepackage{parskip}

\newcommand*{\eqbox}{\tcboxmath[
    enhanced,
    colback=black!10!white,
    colframe=black,
    sharp corners,
    size=fbox,
    boxsep=8pt,
    boxrule=1pt
]}

% Miscellaneous Definitions/Settings
\newcommand{\prob}[2]{\textbf{#1)} #2}

\setlength{\parskip}{\baselineskip}
\setlength{\parindent}{0pt}
\setlength{\headheight}{14.49998pt}
\addtolength{\topmargin}{-2.49998pt}


\begin{document}
    
\prob{1.1}{
If a charged particle is undeflected in passing through uniform crossed electric and magnetic fields $\va*{E}$ and $\va*{B}$ (mutually perpendicular and both perpendicular to the direction of motion), what is its velocity?
If we now turn off the electric field, and the particle moves in an arc of radius $R$, what is its charge-to-mass ratio?
}

We can design our coordinate system such that $\va*{E}=E \vu*{y}$, $\va*{B}=B \vu*{z}$, and $\va*{v} = v \vu*{x}$.
The Lorentz force for a charged particle passing undeflected through these crossed electric and magnetic fields is 
\begin{align}
    \label{eq:undeflected-vel}
    0 &= q(\va*{E} + \va*{v} \cross \va*{B}) = q(E - vB)\vu*{y} \\ 
    &\Rightarrow \eqbox{
        v = \frac{E}{B}
    }   
.\end{align}
Now, if we turn of the electric field, then the electron undergoes uniform circular motion of radius $R$
\begin{align}
    \label{eq:q-to-m}
    m \frac{1}{R}\Bigl( \frac{E}{B} \Bigr)^2 = q\Bigl( \frac{E}{B} \Bigr) B \Rightarrow 
    \eqbox{
        \frac{q}{m} = \frac{E}{R B^2} 
    } 
\end{align}


\prob{1.3}{
In the period before the discovery of the neutron, many people thought that the nucleus consisted of protons and \textit{electrons}, with the atomic number equal to the excess number of protons.
Beta decay seemed to support this idea -- after all, electrons come popping out; doesn't that imply that there were electrons inside? 
Use the position-momentum uncertainty relation, $\Delta x \Delta p \geq \hbar/2$, to estimate the minimum momentum of an electron confined to a nucleus (radius \qty{e-13}{\centi\m}).
From the relativistic energy-momentum relation, $E^2 - \va*{p}^2c^2 = m^2c^{4}$, determine the corresponding energy and compare it with that of an electron emitted in, say, the beta decay of tritium.
}

If we set $\Delta x = \qty{e-13}{\centi\m}$, then the minimum momentum of the electron emitted from the nucleus via beta decay is 
\begin{eqnarray}
    \label{eq:momentum-uncertainty}
    p \approx \Delta p \geq \frac{\hbar}{2 \Delta x} = \SI[per-mode=fraction]{3.3e-7}{\MeV\s\per\m}
.\end{eqnarray}
Plugging this into the energy-momentum relation
\begin{eqnarray}
    \label{eq:energy-from-momentum}
    \eqbox{
        E = \sqrt{\va*{p}^2c^2 + m^2c^4} = \qty{98.7}{\MeV}
    }
.\end{eqnarray}
The typical energy of a beta emission from tritium is approximately \qty{0.019}{\MeV}.
Obviously, our model with the electron being inside the nucleus is incorrect, predicting an emitted energy roughly 5200 times the observed value.


\prob{1.4}{
The \textit{Gell-Mann/Okubo mass formula} relates the masses of members of the baryon octet (ignoring small differences between $p$ and $n$); $\Sigma^{+}$, $\Sigma^{0}$, and $\Sigma^{-}$; and $\Xi^{0}$ and $\Xi^{-}$):
\begin{eqnarray}
    \label{eq:mass-formula}
    2(m_{N} + m_{\Xi}) = 3m_{\Lambda} + m_{\Sigma}
.\end{eqnarray}
Using this formula, together with the known masses of the \textit{nucleon} $N$ (use the average of $p$ and $n$), $\Sigma$ (again, use the average), and $\Xi$ (ditto), `predict' the mass of the $\Lambda$.
How close do you come to the observed value?
}

The \textit{PDG} lists the following masses (in units where $c = 1$) to the nearest integer
\begin{itemize}
    \item $m_{N} = \qty{939}{\MeV}$
    \item $m_{\Xi} = \qty{1318}{\MeV}$
    \item $m_{\Sigma} = \qty{1107}{\MeV}$
\end{itemize}
Hence,
\begin{eqnarray}
    \label{eq:mass-Lambda}
    \eqbox{
    m_{\Lambda} = \frac{1}{3}\Bigl[ 2(m_{N} + m_{\Xi}) - m_{\Sigma} \Bigr] = \qty{1107}{\MeV} 
}
.\end{eqnarray}
The \textit{PDG} lists the mass of $\Lambda$ as \qty{1115}{\mega\eV}, which is only about 1\% larger than our `prediction'.


\prob{1.7}{}

(a) Members of the baryon decuplet typically decay after \qty{e-23}{\s} into a lighter baryon (from the baryon octet) and a meson (from the psuedo-scalar meson octet).
Thus, for example, $\Delta^{++} \rightarrow p^{+} + \pi^{+}$.
List all decay modes of this form for the $\Delta^{-}$, $\Sigma^{*+}$, and $\Xi^{*-}$.
Remember that these decays must conserve charge and strangeness (they are \textit{strong} interactions)

For the following reactions, we must have the baryon from the decuplet decay into one baryon from the baryon octet and a meson from the meson octet.
We also must have decays that respect $S_{\rm baryon} = S'_{\rm baryon} + S_{\rm meson}$ and $Q_{\rm baryon} = Q'_{\rm baryon} + Q_{\rm meson}$, where the unprimed and primed variables with the subscript ``baryon'' represent the original and decay baryons, respectively.

For the $\Delta^{-}$, we must have the total strangeness be equal to $0$ and the charge be $-1$, giving possible reactions as
\begin{align}
    \label{eq:decay-modes-delta}
    \eqbox{
    \Delta^{-} \rightarrow 
    \begin{cases}
    \Sigma^{-} + K^{0} \\
    n + \pi^{-}
    \end{cases}
}
.\end{align}

For the $\Sigma^{*+}$, we must have the total strangeness be equal to $+1$ and the charge be $1$, giving possible reactions as
\begin{eqnarray}
    \label{eq:decay-modes-sig_star_plus}
    \eqbox{
    \Sigma^{*+} \rightarrow 
    \begin{cases}
    \Sigma^{+} + \pi^{0} \\
    \Sigma^{+} + \eta \\
    p + \overline{K}^{0} \\
    \Xi^{0} + \pi^{+} \\
    \Sigma^{0} + \pi^{+} \\
    \Lambda^{0} + \pi^{+}
    \end{cases}
    }    
.\end{eqnarray}

For the $\Xi^{*-}$, we must have the total strangeness be equal to $-2$ and the charge be $-1$, giving possible reactions as 
\begin{eqnarray}
    \label{eq:decay-modes-Xi_star_minus}
    \eqbox{
    \Xi^{*-} \rightarrow 
    \begin{cases}
    \Xi^{0} + \pi^{-} \\
    \Sigma^{0} + K^{-} \\
    \Lambda^{0} + K^{-} \\
    \Xi^{-} + \pi^{0} \\
    \Xi^{-} + \eta \\
    \Sigma^{-} + \overline{K}^{0}
    \end{cases}
}
.\end{eqnarray}


(b) In any decay, there must be sufficient mass in the original particle to cover the masses of the decay products.
Check each of the decays you proposed in part (a) to see which ones meet this criterion.
The others are kinematically forbidden.

For the decays above, we can rule out other forbidden reactions based on conservation of energy, which implies that the $m_{\rm baryon} = m'_{\rm baryon} + m_{\rm meson}$.
Thus, we have the possible decays
\begin{align}
    \label{eq:decay-modes-delta-mass}
    \eqbox{
    \Delta^{-} \rightarrow n + \pi^{-}
}
,\end{align}
for the $\Delta^{-}$,
\begin{eqnarray}
    \label{eq:decay-modes-sig_star_plus-mass}
    \eqbox{
    \Sigma^{*+} \rightarrow 
    \begin{cases}
    \Sigma^{+} + \pi^{0} \\
    \Sigma^{0} + \pi^{+} \\
    \Lambda^{0} + \pi^{+}
    \end{cases}
    }    
\end{eqnarray}
for the $\Sigma^{*+}$, and 
\begin{eqnarray}
    \label{eq:decay-modes-Xi_star_minus-mass}
    \eqbox{
    \Xi^{*-} \rightarrow 
    \begin{cases}
    \Xi^{0} + \pi^{-} \\
    \Xi^{-} + \pi^{0} \\
    \end{cases}
}
\end{eqnarray}
for the $\Xi^{*-}$.

\prob{1.14}{
Using four quarks ($u$, $d$, $s$, and $c$) construct a table of all the possible baryon species.
How many combinations carry a charm of $+1$?
How many carry charm $+2$,
and $+3?$
}

The possible baryon species that can be constructed with four quarks can be determined by listing out all the combinations (with replacement) as in \tref{possible-baryons}.

\begin{table}[H]
\begin{center}
\begin{tabular}{cc}
\toprule
Baryon & charm \\
\midrule
 $uuu$ &     0 \\
 $uud$ &     0 \\
 $uus$ &     0 \\
 $uuc$ &     1 \\
 $udd$ &     0 \\
 $uds$ &     0 \\
 $udc$ &     1 \\
 $uss$ &     0 \\
 $usc$ &     1 \\
 $ucc$ &     2 \\
 $ddd$ &     0 \\
 $dds$ &     0 \\
 $ddc$ &     1 \\
 $dss$ &     0 \\
 $dsc$ &     1 \\
 $dcc$ &     2 \\
 $sss$ &     0 \\
 $ssc$ &     1 \\
 $scc$ &     2 \\
 $ccc$ &     3 \\
\bottomrule
\end{tabular}

\end{center}
\caption{Table of possible quark combinations for baryons ($qqq$), their charge, upness, downess, strangeness, and charmness.}
\label{tab:possible-baryons}
\end{table}

We can see here that there are 6 possible baryons with $+1$ charm, 3 with $+2$ charm, and 1 with $+3$ charm.

\prob{1.15}{
    Same as Problem 1.14, but this time for \textit{mesons}.
}

\begin{table}[H]
\begin{center}
\begin{tabular}{cc}
\toprule
     Meson &  charm \\
\midrule
$u\bar{u}$ &      0 \\
$u\bar{d}$ &      0 \\
$u\bar{s}$ &      0 \\
$u\bar{c}$ &     -1 \\
$d\bar{u}$ &      0 \\
$d\bar{d}$ &      0 \\
$d\bar{s}$ &      0 \\
$d\bar{c}$ &     -1 \\
$s\bar{u}$ &      0 \\
$s\bar{d}$ &      0 \\
$s\bar{s}$ &      0 \\
$s\bar{c}$ &     -1 \\
$c\bar{u}$ &      1 \\
$c\bar{d}$ &      1 \\
$c\bar{s}$ &      1 \\
$c\bar{c}$ &      0 \\
\bottomrule
\end{tabular}

\end{center}
\caption{Table of possible quark combinations for mesons ($q\bar{q}$) with four quarks, their charge, upness, downness, strangeness, and charmness.}
\label{tab:possible-mesons}
\end{table}

We can see from the \tref{possible-mesons} lists $3$ mesons with $+1$ charm content (with 3 corresponding anti-particles carrying $-1$ charm content), but there are no mesons with $+2$ or $+3$ charm.
It is demanded that this happen though since mesons are made of a quark-antiquark pair, so there can be at most one charm. 

\prob{1.17}{
A.~De~Rujula, H.~Georgi, and S.~L.~Glashow estimated the so-called \textit{constituent quark masses} to be $m_{u} = m_{d} = \qty{336}{\mega\eV}$, $m_{s} = \qty{540}{\MeV}$, and $m_{c} = \qty{1500}{\MeV}$.
If they are right, the average binding energy for members of the baryon octet is \qty{-62}{\MeV}.
If they all had \textit{exactly} this binding energy, what would their masses be?
Compare the \textit{actual} values and give the percent error.
}

If we assume that each of the baryons in the baryon octet has the same binding energy of $m_{\rm bind} = \qty{-62}{\MeV}$, then the mass of any given baryon is calculated as follows:
\begin{eqnarray}
    \label{eq:mass-baryon}
    m_{\rm baryon} = n_{u} m_{u} + n_{d} m_{d} + n_{c} m_{c} + m_{\rm bind}
,\end{eqnarray}
where $n_{f}$ is the number of a quark of flavor $f$ in the baryon.
Following this prescription we tabulate the masses below and the percent error of this estimate from the value tabulated in the \textit{PDG}, where the percent error is calculated as
\begin{eqnarray}
    \label{eq:percent error}
    \%~{\rm error} = \frac{{\rm estimate} - {\rm PDG}}{{\rm PDG}} \times 100\%
,\end{eqnarray}
which also indicates whether the estimate calculated here is smaller or larger than the \textit{PDG} value.
From the errors, we see that the estimates are not too bad, being within at least 5\% across the board and within 1\% for the proton and neutron.
\begin{table}[h!tb]
\begin{center}
\begin{tabular}{ccc}
\toprule
     Baryon ($qqq$) & Calculated Mass (\qty{}{\MeV}) & Percent Error from PDG \\
\midrule
          $p~(uud)$ &                            946 &                 0.82\% \\
          $n~(udd)$ &                            946 &                 0.68\% \\
$\Lambda^{0}~(uds)$ &                           1150 &                 3.08\% \\
 $\Sigma^{+}~(uus)$ &                           1150 &                -3.31\% \\
 $\Sigma^{-}~(dds)$ &                           1150 &                -3.96\% \\
 $\Sigma^{0}~(uds)$ &                           1150 &                -3.58\% \\
    $\Xi^{-}~(dss)$ &                           1354 &                 2.44\% \\
    $\Xi^{0}~(uss)$ &                           1354 &                 2.98\% \\
\bottomrule
\end{tabular}

\caption{Calulated masses of particles in baryon octet using \eref{mass-baryon} and the percent difference from the values given by the PDG.}
\label{tab:baryon-mass-table}
\end{center}
\end{table}






\end{document}
